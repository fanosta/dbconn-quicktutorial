\documentclass[11pt, DIV=14]{scrartcl}

\usepackage{ucs}
\usepackage[utf8x]{inputenc}
\usepackage[T1]{fontenc}
\usepackage[ngerman]{babel}
\usepackage{graphicx}
\usepackage{microtype}

\usepackage{amsmath}
\usepackage{lmodern}

% syntax highlighting
\usepackage{minted}
\usepackage{xcolor, colortbl}
\definecolor{gray}{rgb}{0.6,0.6,0.6}
\definecolor{light-gray}{rgb}{0.95,0.95,0.95}
\newcommand{\listingautorefname}{Codesnippet}
\renewcommand{\listingscaption}{Codesnippet}
\renewcommand{\listoflistingscaption}{Codeverzeichnis}
\setminted{
	%numbers=left,
	style=default,
	tabsize=4,
	fontsize=\footnotesize,
	breaklines=true,
	%breakbytokenanywhere=true,
	bgcolor=light-gray
}
\usepackage{float}
\usepackage[hidelinks]{hyperref}
\usepackage{xspace}

\let\java\texttt
\newcommand{\zB}{\mbox{z.\,B.}\xspace}

\title{JDBC-Quicktutorial}
\date{}
\author{}

\begin{document}

        \maketitle


        \section{Klassen für das Abspeichern der Daten}
        \label{sec:data}

        Zum Speichern der Daten wird für jede Tabelle in der SQL-Struktur eine eigenen Java-Klasse definiert.
        Hier wird beispielhaft die Klasse \java{Book} betrachtet.

        \begin{listing}[H]
            \inputminted[firstline=4, lastline=8, gobble=1]{java}{code/book.java}
            \caption{Felder der Book-Klasse}
        \end{listing}

        Für das sinnvolle Verwenden der \java{Book}-Klasse wird noch ein Konstruktor und eine \java{toString()}-Methode benötigt.
        Diese können mit Hilfe der Menüs \textit{Source} $ \rightarrow $ \textit{Generate Constructor using Fields\ldots} und \textit{Source} $ \rightarrow $ \textit{Generate toString()\ldots} in der Menüleiste generiert werden.

        \begin{listing}[H]
            \inputminted[firstline=10, lastline=16, gobble=1]{java}{code/book.java}
            \caption{Konstruktor der Book-Klasse}
        \end{listing}

        \begin{listing}[H]
            \inputminted[firstline=18, lastline=21, gobble=1]{java}{code/book.java}
            \caption{toString() der Book-Klasse}
        \end{listing}

        \section{DBConn für die Datenbankverbindung}
        Die Programmlogik, welche für die Kommunikation mit der Datenbank benötigt wird, wird in der Java-Klasse \java{DBConn} programmiert.

		\subsection{Verbindungsaufbau}
        Im Konstruktor wird eine Verbindung aufgebaut, in den Methoden werden Datenbankabfragen (SELECTs und INSERTs) ausgeführt.

        \begin{listing}[H]
            \inputminted{java}{code/connect.java}
            \caption{Verbindungsaufbau zum MySQL-Server im Konstruktor}
        \end{listing}

		\subsection{Auslesen der Datenbank}
        Wenn eine Tabelle aus der Datenbank abgefragt wird, wird dafür eine eigene Methode geschrieben. In der Methode passieren folgende Dinge:

        \begin{itemize}
            \item Anlegen einer Liste in der das Ergebnis gespeichert wird.
            \item Speichern des SQL-Befehls in einen String
            \item Ausführen des SQL-Befehls mithilfe von \java{stm.executeQuery(sql)}

            \item Durchgehen des \java{ResultSet}s mithilfe von \java{while (rs.next()) \{\ldots\}}
            \begin{itemize}
                \item Erstellen eines neuen Objekts von der zugehörigen Datenklasse (siehe \autoref{sec:data})
                \item Hinzufügen des eben erstellen Objekts zur Ergebnisliste
            \end{itemize}
            \item Returnen der Ergebnisliste
        \end{itemize}

        \begin{listing}[H]
            \inputminted{java}{code/select.java}
            \caption{Abfragen der Tabelle \texttt{books}}
        \end{listing}

		\subsection{Schreiben in die Datenbank}
		Zum Schreiben von Dokumenten in die Datenbank werden auch eigene Methoden definert.
		Hier \zB wird ein Buch in die Datenbank eingefügt.

		\begin{listing}[H]
            \inputminted[firstline=61, lastline=69, gobble=1]{java}{code/DBConn.java}
            \caption{Schreiben in die Tabelle \texttt{books}}
        \end{listing}


		\section{JSPs}

		Will man nun \zB eine Liste aus Büchern (\java{List<Book>}) in tabellarischer Form darstellen kann man folgenden Code verwenden.
		Das Objekt \java{db} wurde hier mit \java{DBConn db = new DBConn();} zu Beginn der JSP erstellt.

		\begin{listing}[H]
			\inputminted[firstline=37, lastline=51, gobble=4]{jsp}{code/index.jsp}
			\caption{Erstellen einer HTML-Tabelle}
		\end{listing}


\end{document}
